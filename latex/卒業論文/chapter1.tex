\chapter{はじめに}
近年,新しい移動手段として乗合タクシーサービスの需要が増加してきている.また,老人や介護が必要な人を自宅などからヘルスケアセンターなどまで輸送するようなサービスも,高齢化に伴い需要が増加してきている.
都心部では、乗合タクシーの実証実験がさかんに行われたりしており、実用化が進んできている.
これらのサービスにおいて、利用者の満足度を考慮しつつサービスの実行にかかるコストを最小化する問題は乗合タクシー問題(Dial-a-Ride problem, DARP)と呼ばれる.
乗合タクシー問題の制約としては,車両の容量資源の最大容量を超えて乗車することはできない(容量制約),車両が回るルートの大きさ(最大ルート距離),顧客が車両に乗っている時間の長さ(最大乗車時間)が与えられる.

本研究では,時間枠及び乗車時間をペナルティ関数で与えてソフト制約とする乗合タクシー問題について考える。ペナルティ関数は区分線形凸関数であるとする。区分線形関数にすることで,時間枠より少しの遅延は許容できる場合など,様々なケースを柔軟に考慮することができるようになり,先行研究より汎用的な問題とすることができる。

3章ではpickup and delivery problem(PDP)を紹介し,乗合タクシー問題(DARP)について詳しく説明する.4章では時間枠及び乗車時間ペナルティ付き乗合タクシー問題の定式化を行う。5章では本問題に対する提案手法を紹介する。
6章では、挿入近傍と交換近傍による解の精度の比較と既存研究の最良解との比較を行った結果を紹介する。また、車両数を減らした際の結果も紹介する。
