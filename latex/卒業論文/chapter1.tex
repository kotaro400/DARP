\chapter{はじめに}
近年,乗合タクシーサービスは新しい移動手段として需要が増加してきている.また,老人や介護が必要な人を自宅などからヘルスケアセンターなどまで輸送するようなサービスも,高齢化に伴い需要が増加してきている.これらのサービスの特徴としては,利用者はリクエストとして出発地と到着地,それぞれに対して場所と時間枠を指定することができる.本研究では,これらのようなサービスにおいて利用者の満足度を考慮しつつサービスの実行にかかるコストを最小化することを考える.このような問題は乗合タクシー問題(Dial-a-Ride problem, DARP)と呼ばれる.

乗合タクシー問題に対しては,多くの研究がなされている.多くの先行研究では,利用者の最大乗車時間とリクエストの乗車時間と降車時間に対しての時間枠をハード制約として与えている.
本研究では,時間枠及び乗車時間をペナルティ関数で与えてソフト制約とする.こうすることで,時間枠より少しの遅延は許容できる場合など,様々なケースを柔軟に考慮することができるようになり,先行研究より汎用的な問題とすることができる.

3章ではpickup and delivery problem(PDP)を紹介し,乗合タクシー問題(DARP)について詳しく説明する.

静的DARPに関して,先行研究において様々な手法が提案されている.Jawらは,この問題に対して近似解法としてルートに挿入した時の目的関数値の増加が最小になるようなリクエストを選択してルートに挿入していく連続挿入法を提案した\cite{insertion}.Cordeauらは,あるルートからリクエストをひとつ取り除き,別のルートに挿入する際にタブーサーチ探索を用いる手法を提案した\cite{tabu}.
