\chapter{提案手法}\label{method}
本節では, 本研究で提案する手法について説明する. 本研究では, リクエストの割り当てと訪問順を局所探索を用いて求める. それぞれの反復で得られたルートに対しては, 乗降時間と乗車時間に関してペナルティ関数の値が閾値以下であるかを判定し,閾値以下であれば最適なサービス開始時刻を決定して評価する.
本研究では制約と目的関数が全て線形の式で表すことが可能であるため,サービス時刻を決定する問題は線形計画問題(linear programming problem, LP)として定式化し,解くことが可能である.
\section{初期解生成}
リクエストをランダムに選び, 車両$k$にリクエストのペアが連続となるようにルートの最後に挿入する. これを$k = 1 から m$まで繰り返した後, 未割り当てのリクエストがあれば$k = 1 $として同様の操作を続ける. これを未割り当てのリクエストがなくなるまで続ける.

\section{制限の緩和}
本研究では, 局所探索を行う上でより自由に探索を行うために, 車両における容量制約を緩和し, 容量制約を破った時のペナルティを計算するペナルティ関数を定義する. 容量制約のペナルティ関数を目的関数に加えた評価関数を用いて解を評価することにより, 実行不可能解も探索可能になる.

車両の容量制約のペナルティは, 車両の最大容量を超えて乗った人数として, $QP$と表す. 車両$k$に対してルートの$i$番目を訪問後に容量を超えて乗っている人数を$QP_i^k$とすると,
\begin{align*}
  QP = \sum_{k \in K}\sum_{i \in n_k} QP_i^k,
\end{align*}
と表せる.
$\gamma$を定数とすると、ペナルティを加えた評価関数を$f(\sigma)$とすると
\begin{align*}
  f(\sigma) = \alpha d(\sigma)+ \beta t(\sigma) + \gamma QP,
\end{align*}
で定義する.
