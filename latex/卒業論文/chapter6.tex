\chapter{計算実験}\label{computational_result}
\section{実験環境}
実験に用いるプログラムはC++を用いて実装し、計算機はプロセッサ 1.4GHz Intel Core i5, メモリ 16GB 2133 MHz LPDDR3のmacOsを搭載したものを使用した。LPソルバーとして、Gurobi Optimizer 9.0.0を使用した。
\section{問題例の作成方法}
DARPでは多くの既存研究があるが、本研究では時間枠及び乗車時間に対して区分線形で凸のペナルティ関数で与えている。このような問題設定のインスタンスは存在していないため、既存研究でよくベンチマークとして使用されているインスタンスに修正を加えて計算実験を行う。修正方法を以下に示す。

時間枠に関しては、サービス開始可能時刻を$e$、サービス開始最遅時間を$l$とすると、0以上$e$以下に対しては傾き-1、$e$から$l$に対しては値が0、$l$以上に対しては傾きが1となるような、区分数が3のペナルティ関数を作成する。この修正作業を全てのリクエストに対して行う。乗車時間に関しては、最大乗車時間を$L$とすると、0以上$L$以下に対しては値が0、$L$以上に対しては傾きが1となるような区分数2のペナルティ関数を作成する。この修正作業を全てのリクエストペアに対して行う。

\section{インスタンスについて}


\begin{figure}[htbp]
 \centering
 \includegraphics[width=0.9\linewidth]{sample_fig.eps}
 \caption{図の表示例}
 \label{fig1}
\end{figure}

\begin{table}[htbp]
 \centering
 \tabcolsep = 15pt
 \renewcommand{\arraystretch}{0.8}
 \caption{表の表示例}
 \label{table1}
 \begin{tabular}{lrr} \hline
  問題例 & 最良値 & 計算時間(秒) \\ \hline
  c05100 &    123 & 10.1 \\
  c10100 &    456 & 15.2 \\
  c20100 &    789 & 20.3 \\ \hline
 \end{tabular}
\end{table}


\begin{table*}
 \centering
 \tabcolsep = 19pt
 \renewcommand{\arraystretch}{0.8}
 \caption{2段組みスタイルにおいて幅の広い表を表示する例}
 \label{table2}
 \begin{tabular}{lrrcrr} \hline
  &\multicolumn{2}{c}{既存手法} && \multicolumn{2}{c}{提案手法}\\ \cline{2-3} \cline{5-6}

  問題例 & 最良値 & 計算時間(秒)&& 最良値 & 計算時間(秒) \\ \hline
  c05100 &    123 &          10.1 &&    111 &          10.0 \\
  c10100 &    456 &          15.2 &&    432 &          15.0 \\
  c20100 &    789 &          20.3 &&    765 &          20.0 \\ \hline
 \end{tabular}
\end{table*}
