\begin{center}
{\LARGE 時間枠及び乗車時間ペナルティ付き\\乗合タクシー問題に対する局所探索法}\\[0.5cm]
\end{center}
\hfill
{\large 051600141\qquad 竹田 陽}\\[0.5cm]
\begin{center}
{\Large \bf 概 要}\\
\end{center}


乗合タクシー問題(diai-a-ride problem, DARP)とは, 異なる移動需要を持つ顧客を同じ車両で同時に輸送する際に, 効率の良い車両の割り当てとスケジューリングを考える問題である. 乗合タクシー問題では, 顧客の乗降点と到着時刻に対する時間枠制約, 乗車時間に対する制約が与えられたときに, 車両の運用にかかるコストおよび顧客の不満度合いを最小にすることを目的とする.

乗合タクシー問題では, 利用者を出発地から目的地まで輸送するサービスにおいて, リクエスト全ての点を訪問すること, 車両はデポから出発しデポに戻ること, リクエストのペアは必ず同じ車両が訪問すること, 出発地より後に到着点を訪問することの全てを満たすルートと車両の割り当てを考える問題である. 利用者は, 出発地と目的地の場所, またその地点に車両が到着してほしい時間枠を指定することができる.
また, 輸送の時間に対して, 最大乗車時間を指定することができる.
乗合タクシー問題に対しては, 多くの研究がなされている. 多くの先行研究では, 利用者の最大乗車時間とリクエストの乗車時刻と降車時刻に対しての時間枠をハード制約として与えている.

時間枠と乗車時間に対する制約を区分線形で凸のペナルティ関数として与えることで, 乗合タクシー問題の顧客の不満度合いをより柔軟に表現できるようにした. その問題を「時間枠及び乗車時間ペナルティ付き乗合タクシー問題」と定義する.

本研究では, 時間枠及び乗車時間ペナルティ付き乗合タクシー問題に対して, ルート内の近傍操作とルート間の近傍操作を用いた局所探索法を用いたアルゴリズムを提案した.
ルート内の近傍操作としては, 挿入近傍と交換近傍の2種類を提案し実装した.
挿入近傍とは, ルートのリクエストを1つ選択し他のルートに挿入する操作のことで, 交換近傍とは, ルートのリクエストを1つ選択し他のルートのリクエストと交換する操作のことである.
計算実験により2種類を比較したところ, 挿入近傍を用いたほうが交換近傍を用いるより良い解を見つけることができることがわかった.

時間枠及び乗車時間のペナルティを大きくすることで, ペナルティの値が小さい解を出力することが可能なので, 先行研究との解の精度の比較を行った. 規模の小さいインスタンスに対しては解の相違が20-30\%程度の解を得ることができたが, 規模の大きいインスタンスに関してはあまり良い結果を得ることができなかった. これは十分な試行回数を行うことができなかったことが理由と考えられる.

また, 本研究では目的関数をルートの長さとペナルティの変数の重み付き和としているため, 係数を変化させた際のルートの比較を行った. 計算結果より, 係数を変化させることでペナルティを優先的に小さくしたい場合や, 少しペナルティが大きくてもルートの長さも小さくしたい場合など, 様々な場合に対応するルートを出力することが可能であることが確認できた.

インスタンスで定められた車両数を1台減らした際の計算実験も行い, 比較を行った. 計算結果より, 車両数を1台減らした際には, 解の精度は少し悪くなり, ペナルティの値も大きくなるが, 実行可能なルートを出力できることがわかった. これにより, サービス自体のコストを下げながら運行を行うことが可能であると考えられる.
